\documentclass{article}
\usepackage[utf8]{inputenc}
\usepackage[a4paper, total={7in, 8in}]{geometry}
\usepackage{amssymb}
\usepackage{amsmath}

\newtheorem{theorem}{Theorem}

\setlength{\parindent}{4em}
\setlength{\parskip}{1em}
\renewcommand{\baselinestretch}{1.5}

\title{Dynamical generalizations of the Prime Number Theorem and disjointness of additive and multiplicative semigroup actions}

\begin{document}

\maketitle

\section{Background and History}

\subsection{Omega Function}

Let $\Omega(n)$ denote the \textbf{number of prime factors} of $n$ (when counted with multiplicities). For example, $\Omega(1)=0,$ $\Omega(p)=1,$ $\Omega(p.q)=\Omega(p^2)=2,$ $\Omega(p_1^{e_1}\cdots p_k^{e_k})=e_1 + \dots + e_k.$

A central question that we have in number theory is: What is the distribution of the values of $\Omega(n).$ If we look at the values of the function for the first two-hundred positive integers, we find that $\Omega(n)$ is growing. In fact, it grows like $\log(\log(n)),$ but besides the average growth, it behaves very randomly. We shouldn't forget here that there are loads of primes within the first two-hundred integers. So, it's understood that:

\begin{itemize}
    \item The distribution of the values of $\Omega(n)$ follows no notable pattern. It appears to be random.
    \item Knowing $\Omega(n-1),$ $\Omega(n-2),\cdots, \Omega(n-m)$ does not allow us to predict $\Omega(n).$
\end{itemize}

\subsection{Some Classical Results in Multiplicative Number Theory}

The study of the distribution of the values of $\Omega(n)$ has a long and rich history and is closely related to fundamental questions about the prime numbers.

The \textbf{natural density} of a set $A \subset \mathbb{N}$ is defined as $d(A)=\lim_{n \to \infty}|\{1 \le n \le N : n \in A\}|/N.$ The following statement is a well-known equivalent form of the Prime Number Theorem (von Mangoldt 1897, Landau 1911): 

\begin{theorem}
The sets $\{n \in \mathbb{N} : \Omega(n)$ is even$\}$ and $\{n \in \mathbb{N} : \Omega(n)$ is odd$\}$ have natural density of $1/2.$
\end{theorem}

This means that, asymptotically, there are as many prime numbers with even numbers of prime factors as there are prime numbers with odd numbers of prime factors. It is natural to ask whether this can be generalized as this function is evenly distributing between even's and odd's. And we can see this in the Pillai-Selberg Theorem (Pillai 1940, Selberg 1939):

\begin{theorem}
For all $m \in \mathbb{N}$ and $r \in \{0,\dots,m-1\}$ the set $\{n \in \mathbb{N} : \Omega(n) \equiv r \mod m\}$ has natural density $1/m.$
\end{theorem}

<https://youtu.be/2n0elpGbvYU?t=325>

\end{document}
